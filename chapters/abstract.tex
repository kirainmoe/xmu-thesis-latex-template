% 摘要(中文)

\xmusetup{
  keyword = 非结构化信息;信息可视化;可视分析
}

\begin{abstract}

随着信息的发展,出现了越来越多的非结构化信息。并且非结构化信息在政府和企业等的决策中扮演着重要的角色。如何将非结构化数据有效的管理起来,能够进行数据和知识挖掘,提取当中的隐含信息,提供一种形象的可视分析,为政府和企业决策提供支持成为当今亟待解决的主要问题。

本文以北京市科委的指数统计文档为研究对象,主要任务是针对以北京市科委的指数统计文档为代表的非结构化信息的抽取和企业指标信息的可视分析。主要工作包括三个方面:第一,设计了一套以北京市科委的指数统计文档编写规范为标准的确实可行的信息抽取算法;第二,针对抽取出来的指标信息,借助于Dundas可视化工具进行可视分析;第三,完成了一个满足客户需求的企业信息库管理系统。

论文从项目背景出发,介绍了系统开发的背景和研究价值。然后,详细介绍了企业指标信息智能处理的可行性和算法设计,以及企业指标信息可视分析的原理及其实现。再次,论文详细阐述了系统的需求,具体介绍了企业信息库管理系统的设计及其实现,最后论文针对企业信息库管理系统进行了分析和评价,并指明了下一步的改进计划。\footnote{这是一个注解示例。}

\end{abstract}