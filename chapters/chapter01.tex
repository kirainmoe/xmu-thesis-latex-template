\newchapter{绪论}{Introduction}

\newsection{引言}{Preface}

随着计算机技术的发展,使海量信息得以存在并迅猛发展。尤其是信息技术的日益普及其应用以后,随着各个行业的信息系统的规模的日益扩大,信息系统在长年累月的运转过程中,积累了庞大的数据资源。然而决策者却很难利用这些数据资源,为企业和政府的决策提供确实有效的帮助。这是因为一方面,在这庞大的数据资源中,非结构化信息占据了主要部分\cite{fsyang-skgao-1998}。 Gartner的一项调查显示,在今天的社会中,有80\% 以上的商业行为依赖于非结构化信息;我们所存储的数据中,85\%以上是非结构化信息;每过三个月,我们周围的非结构化信息就会增加一倍。这些数据充分说明,我们周围信息的形态是以非结构化信息为绝对主体的,也可以说我们接触到的信息中绝大部分是非结构化信息。因此对非结构化信息进行管理,能够进行数据和知识挖掘,提取当中的隐含信息,对决策进行支持成为当今亟待解决的主要问题 \cite{cbchen2008}。另一方面,随着信息技术的发展,信息结构越来越复杂,信息更新越来越快,信息规模越来越大,给人们获取信息、理解信息、掌握信息带来了沉重的负担,常常导致“认知过载”、“视而不见”\cite{dzzhang-ppzhang-2006}\cite{xdxie-1998}。

北京市科学技术委员会在企业指标信息统计分析工作上就存在这两方面的问题,文献\parencite{xdxie-1998}介绍了这方面的工作。每年北京市科委都要对北京市企业进行企业指标信息的调查,在长年累月的积累过程中,北京市科委积累了大量的企业指标调查表、项目立项、执行、验收等文档。这些调查表以word形式保存起来,并且调查指标的方式也呈现多样化,存在选择、填空、表格、问答以及这些题目的复合等形式。而且企业指标的调查涵盖范围也很广泛,从企业性质及登记情况到企业财务及信息化投入状况,再到人力状况及信息化支撑状况,到企业信息化基础设施建设状况、企业信息化应用情况,甚至涉及到企业对信息化工程的满意程度的调查。面对海量的非结构化企业指标信息,北京市科委每年都要投入大量的人力、物力、精力,将企业指标信息从word文档中手工提取出来,形成计算机可以识别的结构化的表格信息,再对企业指标信息进行统计分析。即使是这样,仍然存在许多问题:第一,手工抽取企业信息调查表耗时较长,工作强度大。第二,手工抽取数据信息容易出现错误,准确性不能得到有效保证,而且一旦出错,就有可能导致整个统计分析结果的错误,进行核对非常困难。第三,即使是将企业指标信息全部准确转成计算机可以识别的表格数据以后,由于数据的多样性,缺少形象的对企业指标信息的统计分析工具。 

针对北京市科委的企业指标信息统计分析问题,我的毕业设计结合北京市科委的业务需求,开发了企业信息库管理系统。这个项目来源于国家科技支撑计划项目课题“面向服务的智能化制造技术及示范应用”(课题编号2006BAF01A17)。该项目主要是为了解决北京市科委的指标信息统计分析过程中,存在指数统计困难和文档管理困难两个问题,以业务为主线,主要包括科委文档的管理、企业指标信息的智能处理、企业指标信息的可视分析三个方面的内容。通过为科委中存在的大量信息文档实体构建基础信息模型,来方便用户的日常管理和提高文档的利用率。通过构建应用数据模型,将企业指标信息文档中的非结构化信息智能抽取出来,并存储于数据库当中,将非结构化信息结构化,用成熟的结构化数据管理理论来管理非结构化数据。通过对指标信息的查询,构建信息可视分析模型,使用户可以对知识进行挖掘,提供形象的可视分析,提高北京市科委的企业指标信息的统计分析效率。本项目完成后将会在北京市科委投入使用。

\newsection{论文组织结构}{Organization of Thesis}

本论文共分为六章,论文首先分析了政府和企业在信息化过程中遇到的两个问题:非结构化信息管理和“认知过载”。并结合北京市科委的企业指标统计分析问题,介绍了毕业设计项目的背景和研究价值,引出了论文所做的主要工作内容。紧接着论文简单概述了毕业设计项目中所用到的各项技术,并针对北京市科委的业务要求提出了信息抽取和基于Dundas Chart信息可视化的解决方案。然后论文就项目中的两个技术难点——非结构化信息处理和信息可视分析,详细阐述了信息抽取技术的算法设计和信息可视分析技术的模型设计。在系统实现方面,论文详细介绍了企业信息库管理系统所使用的技术要点:基于Asp.net的三层结构(USL-BLL-DAL)的框架设计;在用户体验方面,采用了Asp.net Ajax改善用户的体验 \cite{ll-soidc}。

论文具体安排如下:

第一章\ 简单介绍了企业和政府在信息化过程中遇到的非结构化信息管理困难和“认知过载”问题。针对北京市科委的指标统计分析问题,提出了毕业设计的背景、目标和研究价值。

第二章\ 概述系统中所使用的各项技术及各项技术的国内外发展现状。

第三章\ 详细介绍了针对北京市科委企业指标信息文档的信息抽取技术的算法设计和信息可视分析的模型设计。

第四章\ 介绍了企业信息库管理系统的实现。详细阐述了系统的背景和总体目标,基于表示层(USL)-业务逻辑层(BLL)-数据访问层(DAL)的三层结构的框架设计和功能模块介绍及其实现。

第五章\ 介绍了企业信息库管理系统的系统测试和运行结果。

第六章\ 最后论文总结了毕业设计所做的工作,并且指明了下一步的改进计划。主要是在信息抽取算法的改进,以及在用户体验方面的改进计划。
